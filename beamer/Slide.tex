\documentclass[11pt,compress,xcolor=x11names,UTF8]{beamer}
\usetheme{Singapore}
\usefonttheme{structurebold}
\usecolortheme[named=SpringGreen4]{structure}
\usepackage{gradientframe} % 图片立体感
\graphicspath{{figure/}} % 图片路径
\usepackage{calligra} % Thank you
\usepackage{ctex} 


\useinnertheme{circles}
\usepackage{pifont}
\usepackage{academicons}
\usepackage{fontawesome}
\usepackage{iitem}
\usepackage{amsmath}
\setbeamertemplate{itemize item}{\ding{108}}
\setbeamertemplate{itemize subitem}{\ding{109}}
\setbeamertemplate{navigation symbols}{}
\setbeamercovered{transparent}  
\renewcommand\appendixname{附录}
\renewcommand\abstractname{摘要}
\graphicspath{{figure/}} % 图片路径
\usepackage{calligra} % Thank you
\usepackage{ctex} % 加入中文

\usepackage{url}					
\usepackage{setspace}
\renewcommand{\baselinestretch}{1.2}
\title{环境管制降低了污染还是转移了污染?}
\subtitle{——基于《大气污染防治重点城市划定方案》的实证研究}
\author[罗文杰]{学生:罗文杰 \and 导师:向训勇 } 
\institute[暨南大学]{产业经济研究院 \and 2019级硕士开题答辩} 
\date{2020/11/11}

\begin{document}

\maketitle

\begin{frame}{Outline}
\tableofcontents
\end{frame}

\section{文章内容}

\subsection{政策评价}

	\begin{frame}{政策评价}
		\textbf{《大气污染防治重点城市》}
		\begin{itemize}
			\item 2002年12月,在《大气污染防治法》基础上制定
			\item 选取了113个重点防治城市
			\item 要求在2005年达到国家环境大气质量标准
			\begin{itemize}
				\item 二氧化硫、二氧化氮、悬浮颗粒物、可吸入颗粒物
			\end{itemize}
		\end{itemize}
	\end{frame}

	\begin{frame}{污染降低还是污染替代?}
		\textbf{企业如何减少排污?}
	\begin{itemize}
		\item change-in-process
		\begin{itemize}
			\item 减少生产量
			\item 购买更清洁的生产设备
		\end{itemize}
	
		\item end-of-pipe
		\begin{itemize}
			\item 沉渣室和气体洗涤器
			\item 气体污染转移到地面、水中
		\end{itemize}
	\end{itemize}
	\end{frame}



\subsection{文献综述}
	\begin{frame}{文献综述}
		\begin{itemize}
			\item 环境管制的有效性
			\begin{itemize}
				\item 包群, 邵敏, 杨大利, 2013. 环境管制抑制了污染排放吗?[J]. 经济研究, 12: 42-54.
				\item {\rmfamily Greenstone M, 2004. Did the Clean Air Act cause the remarkable decline in sulfur dioxide concentrations?[J]. Journal of Environmental Economics and Management, 47(3): 585-611.}
			\end{itemize}
			
			\item 环境管制引致的污染替代
			\begin{itemize}
				\item  {\rmfamily Greenstone M, 2003. Estimating regulation-induced substitution: The effect of the Clean Air Act on water and ground pollution[J]. American Economic Review, 93(2): 442-448.}
				\item  {\rmfamily Gibson M, 2019. Regulation-induced pollution substitution[J]. Review of Economics and Statistics, 101(5): 827-840.}
			\end{itemize}
		\end{itemize}
	\end{frame}
\subsection{边际创新}
	\begin{frame}{边际创新}
		\begin{enumerate}
			\item 从``数''和``量''两方面评价政策
			\begin{itemize}
				\item ``数'':环境管制有没有降低污染
				\item ``量'':企业通过什么方式降低污染
			\end{itemize}
			\item 较新的企业层面数据
			\item 识别微观企业的应对策略
			\begin{itemize}
				\item 污染排放的主体
				\item 地区和行业层面的数据
				\item 考察企业的异质性
			\end{itemize}
		\end{enumerate}

	\end{frame}

\section{计量模型}

\subsection{DID}
	\begin{frame}{计量方程}
	\begin{equation}
		Pollutant_{it} =\alpha_{i} + \gamma_{t} + \beta_1  Treat_{i} \times Post2003_{t} + X _{it}'\delta  + \epsilon_{it}
	\end{equation}
	\begin{itemize}
		\item 下标$i$表示企业,下标$t$表示年份
		\item $Pollutant_{it}$ 为企业$i$在$t$年排放的污染物量
		\item 若企业$i$所在城市为大气污染防治重点城市,则$Treat_{i} $为1,否则为0
		\item 2003年及以后$Post2003_{t}$为1,2003年以前为0
		\item $X _{it}'$为企业层面的控制变量
		\item $\alpha_{i} $和$ \gamma_{t} $分别为个体固定效应和时间固定效应
	\end{itemize}

	\end{frame}

\subsection{企业排污}
\begin{frame}{初步结果}{初步结果-空气污染物}

	
	\begin{tabular}{l*{4}{c}}
		\hline\hline
		&\multicolumn{1}{c}{(1)}&\multicolumn{1}{c}{(2)}&\multicolumn{1}{c}{(3)}&\multicolumn{1}{c}{(4)}\\
		&\multicolumn{1}{c}{ln($SO_2$)}&\multicolumn{1}{c}{ln(煤炭消费)}&\multicolumn{1}{c}{ln(烟尘)}&\multicolumn{1}{c}{ln(工业粉尘)}\\
		\hline
		$Treat_{i} \times$      &   -0.228***&   -0.230***&   -0.181***&  -0.0477*  \\
		 $Post2003_{t}$ &  (-7.45)   &  (-9.91)   &  (-5.30)   &  (-1.67)   \\
		\hline
		控制变量       &   Y&   Y&   Y&  Y \\
        固定效应                 & Y  & Y   &  Y   &  Y  \\
       
		\hline
		\(N\)     &   206660   &   161785   &   201432   &   176259   \\
		\(R^{2}\) &    0.849   &    0.908   &    0.820   &    0.901   \\
		\hline\hline
	\end{tabular}
	
\end{frame}

\begin{frame}{初步结果-水污染物}
	
	
\begin{tabular}{l*{3}{c}}
	\hline\hline
	&\multicolumn{1}{c}{(1)}&\multicolumn{1}{c}{(2)}&\multicolumn{1}{c}{(3)}\\
	&\multicolumn{1}{c}{ln(氨氮排放量)}&\multicolumn{1}{c}{ln(COD排放量)}&\multicolumn{1}{c}{ln(废水排放量)}\\
	\hline
	$Treat_{i} \times$      &    0.322***&  -0.0610*  &  -0.0120   \\
	 $Post2003_{t}$ &   (9.78)   &  (-1.82)   &  (-0.39)   \\
		\hline
控制变量       &   Y&   Y&    Y \\
固定效应                 & Y  & Y   &  Y   \\
\hline
	\hline
	\(N\)     &   148298   &   206369   &   209487   \\
	\(R^{2}\) &    0.752   &    0.805   &    0.814   \\
	\hline\hline

\end{tabular}
	
\end{frame}

\subsection{污染替代}
\begin{frame}{污染替代}
	Gibson M, 2019. Regulation-induced pollution substitution[J]. Review of Economics and Statistics, 101(5): 827-840.
	\begin{equation}
	ln(\tfrac{AirPollutant_{it}}{WaterPollutant_{it}}) =\alpha_{i} + \gamma_{t} + \beta_1  Treat_{i} \times Post2003_{t} + X _{it}'\delta  + \epsilon_{it}
	\end{equation}  
\begin{tabular}{l*{4}{c}}
	\hline\hline
	&\multicolumn{1}{c}{(1)}&\multicolumn{1}{c}{(2)}&\multicolumn{1}{c}{(3)}&\multicolumn{1}{c}{(4)}\\
	&\multicolumn{1}{c}{$SO_2$/废水}&\multicolumn{1}{c}{煤炭/废水}&\multicolumn{1}{c}{烟尘/废水}&\multicolumn{1}{c}{粉尘/废水}\\
	\hline
	$Treat_{i} \times$       &  -0.0325***&-0.000585   &  -0.0281***& -0.00757   \\
	$Post2003_{t}$ &  (-4.06)   &  (-0.20)   &  (-4.02)   &  (-1.22)   \\
		\hline
控制变量       &   Y&   Y&   Y&  Y \\
固定效应                 & Y  & Y   &  Y   &  Y  \\
	\hline
	\(N\)     &   170039   &   131947   &   167396   &   144242   \\
	\(R^{2}\) &    0.774   &    0.785   &    0.739   &    0.859   \\
	\hline\hline

\end{tabular}
\end{frame}


\section{未来的工作计划}
\subsection{政策的内生性}
	\begin{frame}{PSM匹配}
		\textbf{政策制定和要求}
		\begin{itemize}
		\item 全国有大气环境质量监测数据的338个城市
		\item 城市总体规划、环境保护规划目标和城市大气环境质量状况
		\item 综合经济能力和环境污染现状
		\end{itemize}
	\end{frame}
\subsection{机制检验}
	\begin{frame}{机制检验}
		\begin{itemize}
			\item 企业间
			\begin{itemize}
				\item 企业进入与退出
			\end{itemize}
			\item 企业内
			\begin{itemize}
				\item 污染处理
				\item 经济特征
			\end{itemize}
			
		\end{itemize}
		

	\end{frame}
\subsection{稳健性与异质性检验}

	\begin{frame}{稳健性与异质性检验}
		\textbf{稳健性}
		\begin{itemize}
			\item DID平行趋势检验
			\begin{itemize}
				\item 处理组和控制组的污染物排放量
				\item 政策的动态效应
			\end{itemize}
		\end{itemize}
		\textbf{异质性}
		\begin{itemize}
			\item 行业层面
			\begin{itemize}
				\item 污染行业和非污染行业
				\item 空气污染行业和非空气污染行业
			\end{itemize}
		\item 企业层面
		\begin{itemize}
			\item 国有企业和非国有企业
		\end{itemize}
		\end{itemize}
	\end{frame}


	\begin{frame}
	\centering {\zihao{0}  \calligra{Thank You}}
	\end{frame}





\end{document} 


