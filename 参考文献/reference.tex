\documentclass{ctexart}

\title{}
\author{罗文杰 \\ 1930071013 \\ 暨南大学产业经济研究院}
\date{}

\usepackage{graphicx}
\usepackage{graphics}
\usepackage{multirow}
\usepackage[colorlinks=true,linkcolor=red,citecolor=black]{hyperref}
\usepackage{geometry}
\geometry{a4paper,scale=0.9,left=3.18cm,right=3.18cm,top=2.54cm,bottom=2.54cm}
\renewcommand{\thefootnote}{\fnsymbol{footnote}}
\renewcommand{\baselinestretch}{1.6}

%\usepackage[backend=biber,style=chinese-erj]{biblatex}
\usepackage[backend=biber,style=gb7714-2015ay]{biblatex}

\addbibresource[location=local]{reference.bib}

\usepackage{placeins}


\begin{document}
	\maketitle
	\section{研究背景与目的}
中国经济的快速增长伴随着环境污染的快速增长,\textcite{world2007cost}的报告指出中国500个城市中,空气质量符合世卫组织推荐标准的不足1\%; 
环境的高污染还给经济发展带来了巨大成本,环境保护部和国家统计局发布的《中国环境经济核算报告》指出,基于环境退化成本的环境污染代价从2004年的5118.2亿元提高到2009年的9701.1亿元。

为了应对大气污染问题,中国政府在1987年制定了《大气污染防治法》,并于1995年、2000年、2015年和2018年进行了修订。2002年12月,根据2000年修订的《大气污染防治法》第十七条第一款“国务院按照城市总体规划、环境保护规划目标和城市大气环境质量状况,划定大气污染防治重点城市”,国家环保总局发布了《大气污染防治重点城市规定方案》。

《大气污染防治重点城市规定方案》划定了113个重点城市,除43个直辖市、省会城市、沿海开放城市和重点旅游城市外,还根据城市综合经济及环境污染现状以及《酸雨和二氧化硫污染防治“十五”计划》中的相关要求(2005年达标的地级城市、目前大气环境质量超标但有望在2005年达标的城市和一些急需加强保护的文化、旅游城市),选定了其他70个城市划定为大气污染防治重点城市。
《大气污染防治重点城市规定方案》要求这113个城市在2005年时,城市市区的大气二氧化硫、二氧化氮、总悬浮颗粒物和可吸入颗粒物浓度稳定保持在相应的大气环境质量标准内。

企业可以通过``change-in-process''和``end-of-pipe''两种方式减少排污\cite{liu2018corporate},前者指的是企业在生产过程中减少污染的产生,实现的途径包括减少生产量、购买更清洁的生产设备;后者指的是企业在污染产生后对污染进行处理,减少污染物的排放量,实现的途径包括加装净化设备。在处理气态污染物(如二氧化硫、PM)时,``end-of-pipe''多使用沉渣室和气体洗涤器,这些设备对污染物的去除率达到90\%--99\%,但这些设备并没有真正去除污染物,而是把污染物转换为了固定或废水,并排放到了水中\cite{greenstone2003estimating}。
然而,中国的水污染也不容乐观:\textcite{world2006china}的报告指出中国500个河流监测断面中,33\%的水体受到严重污染,不适用于任何用途。

本文丰富了环境管制对微观企业影响的研究。中国工业企业数据库对研究者开放申请后,已经有大量文献研究了环境管制对企业经济特征的影响,如投入、产出、全要素生产率以及就业、出口等。目前,仍较少有文献定量研究环境管制对企业污染物排放多少的程度以及企业为应对环境管制做出的策略性调整,本文使用环保部的环境调查报告数据库,识别企业的减排措施,

本文还提供了评价环境管制政策的新视角。以往的研究多集中研究环境管制政策制定后,对社会经济、福利的影响(如\textcite{muller2009efficient}),而忽略了政策本身的有效性及其隐性影响。本文通过在微观层面探讨企业的污染物减排量和管制导致的污染物排放媒介转移,评价在政策制定后企业的应对措施,为政策更合理地制定提出了建议。


\section{文献综述}
\subsection{环境管制的假说}
关于环境管制对微观企业影响的途径、机制,可分为污染天堂假说(如\textcite{mcguire1982regulation})和波特假说\cite{porter1995green,porter1995toward},它们从企业竞争的角度分析了企业为应对管制做出的调整,讨论了环境管制的利弊。
\subsubsection{污染天堂假说}
污染天堂假说认为在其他企业特征一致的情况下,受到更严厉的环境管制的企业会失去竞争力,企业会转移到环境管制相对不严厉的地方(即污染天堂)进行生产。

其核心依据为,环境管制给企业的运作带来了高成本,挤出了提高创新和效率的生产性投资,降低了产能增长。在竞争激烈的产品市场上,如果企业为应对管制带来的成本传递到了产品价格上,那么环境管制越严格的地方,产品价格也会越高,进而造成了贸易的错配。

从竞争的角度考虑,在环境管制严格的地方生产、销售的企业,将不如在环境管制不严格的地方生产、在环境管制严格的地方销售的企业有竞争力。如果企业预期到环境管制将长时间存在,那么在生产新产品,尤其是污染密集型产品时,将会考虑选择管制政策更松的生产地,该生产地即会成为“污染天堂”\cite{dechezlepretre2017impacts}。

污染天堂假说对环境管制的有效性提出了质疑,因为环境管制并没有实质上减少污染,而是把污染转移到了环境政策宽松的地方。
\subsubsection{波特假说}
波特假说与污染天堂假说相反,它认为更严厉的环境管制会使企业更多地投资于减排技术,有利于提高企业竞争力。

其核心论据为,环境管制导致企业投入成本创新,采用更清洁的技术带来了更高的产出、更低的投入成本;在长期,除了抵消管制成本外,还提高了企业出口和市场份额,最终提高了企业竞争力。由于存在学习的外部性,一个企业不会先于它的竞争对手使用更新的、更清洁的生产技术,而环境管制迫使企业做出改变,实现了环境保护和经济增长的“双赢”\cite{王兵2008环境管制与全要素生产率增长}。

波特假说支持了环境管制的有效性,\textcite{porter1995toward}提出一个国家可以通过比另一个国家更早地制定环境管制政策,提高本国企业的清洁技术在全球市场上的需求,从而具有国际竞争力。

\subsection{环境管制的实证研究}
\subsubsection{环境管制的有效性}
以研究广泛的美国清洁空气法案修正案(CAAA,Clean Air Act Amendments)为例,\textcite{greenstone2004did}运用了1975-1980、1981-1986和1987-1992三个时间段的县级二氧化硫监测数据,发现更严格的空气管制对二氧化硫排放量的减小影响微乎其微;\textcite{auffhammer2009measuring}发现在受到管制的县,CAAA对其1990年到2005年的PM10监测器读数没有影响。\textcite{auffhammer2011clearing}进一步考察了环境管制的异质性,通过研究美国汽油清洁法案在不同地区的严格程度,发现当企业应对管制所需的成本较小时,环境管制对当地的臭氧量没有影响,只有实施严厉管制法案的州的空气质量才有显著提高。\textcite{greenstone2014environmental}以印度的环境管制为背景,发现空气污染管制有效降低了颗粒污染物、二氧化硫和氮氧化物的排放,但是水污染管制法案没能有效提升水的质量。

研究中国环境管制政策有效性的文献结论并不一致,如\textcite{包群2013环境管制抑制了污染排放吗}以1990年以来的84件地方人大的环保立法为案例,研究地方环境管制是否降低了污染排放,发现仅仅是立法本身并不能有效抑制污染排放,相反,只有立法后执法力度严格的省份的污染物排放降低了。
\textcite{zhang2018does}使用2006至2009年的企业污染物排放数据,研究了中央督察对当地企业排污的影响,通过逐年断点回归,发现在政策实施一年后,企业的化学需氧量(COD)排放量显著下降了26.8\%,证实了更严格的环境管制有利于降低污染物的排放。
使用1998--2012年中国省级面板数据,\textcite{余长林2015环境管制对中国环境污染的影响}以隐性经济的视角研究了环境管制对排污的影响,发现环境管制通过扩大隐性经济规模提高了环境污染,认为政府应制定环境管制时应注意调整强度政策强度,对企业约束的同时也起到激励作用。
\subsubsection{环境管制对微观企业的影响}
在验证污染天堂假说和波特假说时,已经有较多的文献从全要素生产率的视角研究了环境管制对企业的影响,这是因为企业全要素生产率的变化既能体现企业生产技术的变化,也能较好体现经济发展的质量\cite{王杰2014环境规制与企业全要素生产率}。一类文献认为环境管制降低了企业的全要素生产率,机制是由于环境管制发生后,不论使用''change-in-process''还是''end-of-pipe''方式应对,企业通常会加装节能减排的设备,从而增加了生产成本,降低了收入生产率,因此环境管制对企业的生产率有着负面作用。\textcite{gollop1983environmental}使用美国1973-1979年电力行业的企业数据,通过排污标准、执法力度衡量管制强度,发现污染管制显著提高了企业生产成本,生产率每年下降约0.56\%。\textcite{wang2018environmental}发现中国的水污染管制显著降低了企业的产出和生产率,在现有生产技术下,每减少10\%化学需氧量的排放就会降低0.1\%的产出。\textcite{盛丹2019两控区环境管制与企业全要素生产率增长}以中国两控区环境管制为准自然实验,发现环境管制提升了企业的生产成本,阻碍了全要素生产率的提高。

另一类文献则认为环境管制提高了企业的全要素生产率,即验证波特假说。\textcite{hamamoto2006environmental}使用1971-1986年日本的行业数据,发现环境管制强度对R\&D行为有正向影响,同时增长的的R\&D投资促进了企业全要素生产率的提高。\textcite{berman2001environmental}以1979-1992年美国的环境管制为例,发现管制虽然给企业带来了高成本,但是减排措施提高了生产力。

除生产率外,文献也研究了环境管制对企业其他经营行为的影响,如劳动力雇佣\cite{berman2001_2,walker2011environmental}、国际贸易\cite{levinson2008unmasking,aichele2015kyoto,hanna2010us}、进入退出市场\cite{deily1991enforcement,becker2000effects,wang2018environmental,he2002urban,盛丹2019两控区环境管制与企业全要素生产率增长}。本文将以上述文献的研究为基础,进一步探讨环境管制的影响机制。

不少文献研究了环境管制对企业影响的异质性。如不同行业的企业\cite{李树2013环境管制与生产率增长,he2020watering}、不同地区的企业\cite{王兵2010中国区域环境效率与环境全要素生产率增长,李虹2018环境规制}和不同所有制的企业\cite{沈坤荣2017环境规制引起了污染就近转移吗,杜龙政2019环境规制}。本文将在以上文献的研究基础上,进一步分析环境管制的异质性影响。

本文还参考了\textcite{GREENSTONE200921}的建议完善计量模型。
\subsubsection{环境管制引致的污染替代}
为了应对管制,企业内部可能通过改变生产过程,如果把污染物看作是一种投入,那么企业会减少管制所需的污染物的投入,增加另一种不受管制污染物的投入,进而导致了污染替代\cite{gibson2019regulation,holland2011spillovers};另一种可能的情况是,企业为了以最小成本应对环境管制,选择加装``end-of-pipe''的方式处理已经产生的污染物,进而造成空气污染物向地表或水污染物转移\cite{greenstone2003estimating}。

实证研究中,受限于微观数据的可获得性,环境管制是否造成了企业内部的污染替代仍较少有研究,\textcite{greenstone2003estimating}运用钢铁行业的数据研究了CAAA对污染替代的影响,没有发现空气污染向水和地表转移,而\textcite{gibson2019regulation}的研究则发现受CAAA管制的企业的水污染提高了105\%,未受到管制的企业的空气污染提高了11\%。\textcite{holland2011spillovers}讨论了空气污染物之间的替代效应,使用更清洁的能源(电、天然气等)不仅降低了二氧化硫的排放,还降低了氮氧化物的排放。

目前研究污染替代的文献集中于发达国家,作为世界上最大的发展中国家,中国的产业结构、能源使用以及环境管制力度可能与发达国家不同,本文填补了当前研究的空白。

	\FloatBarrier
	\newpage
    \printbibliography[heading=bibliography,title=参考文献]
	
\end{document}